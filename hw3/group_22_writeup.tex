\documentclass[letterpaper,10pt,notitlepage,fleqn]{article}

\usepackage{nopageno} %gets rid of page numbers
\usepackage{alltt}                                           
\usepackage{float}
\usepackage{color}
\usepackage{url}
\usepackage{balance}
\usepackage[TABBOTCAP, tight]{subfigure}
\usepackage{enumitem}
\usepackage{pstricks, pst-node}
\usepackage{geometry}
\geometry{textheight=9in, textwidth=6.5in} %sets 1" margins 
\newcommand{\cred}[1]{{\color{red}#1}} %command to change font to red
\newcommand{\cblue}[1]{{\color{blue}#1}} % ...blue
\usepackage{hyperref}
\usepackage{textcomp}
\usepackage{listings}

\def\name{Group 22}

\parindent = 0.0 in
\parskip = 0.2 in

\title{Project 2 Write Up}
\author{Group 22}

\begin{document}
\maketitle
\hrule

\section*{Our Plan}
    To create an encrypted \textit{RAM Disk} we decided to first get a unencrypted 
    version working before we begin to work with the \textit{Linux Crypto API}. 
    After reading over the assignment details and requirements we took a look at 
    the ``Linux Device Drivers'' example we discovered that finding a working example 
    of a \textit{RAM Disk} would not be too hard to find. After hearing about how 
    undocumented the \textit{Linux Crypto API} is in class we will begin our research 
    early. 

\section*{Our Solution}
    We took advantage of reusing the code from the ``Linux Device Drivers'' book that 
    was reccomneded for us to get used to programming Linux Drivers. We wanted to get 
    a working \textit{RAM Disk} working before we impplemented our \textit{cipher}. 
    We added the sbd.c text that we grabbed from the book and after a few compile warnings 
    we found a blog by ``Pat Patterson'' that said he had fixed the warnings and errors 
    from the sbd.c file from the ``Linux Device Drivers''. Once we compiled the kernel with 
    the ramdisk included it was we needed to mount it. Using the command \textit{fdisk -l} 
    we descoverd that our new \textit{RAM Disk} was systemly mapped to \textit{/dev/sbd0} 
    since the \textit{RAM Disk} is just a unformated block of memory we decided to make a 
    ext2 filesystem and localy mounted it to a folder. After the Disk was mounted we copied 
    a few files to it and 
\section*{Work Log}

\begin{center}
    \begin{tabular}{| p{3cm} | l | l | p{5cm} |}
        \hline
        Date & Author & Commit & Summary \\ \hline
        Tue Oct 14 15:00 & Sam Quinn & 9402a8ddf13ad69b80eb9bf42c294822aff87b2f & Added the Linux folder for homework \#2.
        \\ \hline
        Fri Oct 17 13:39 & Bob & 8daeb619e3ebf201b079b3643e83c70c119f8de0 & Coppied the Noop I/) scheduler for a template in creating the new SSTF scheduler.
        \\ \hline
        Wed Oct 22 18:42 & Bob & 6fbc561c60d26ea444b063b912ac01068f5fca44 & Implemented sstf\_dispatch
        \\ \hline
        Sat Oct 25 15:22 & Bob & 3c4249ea31657ed2d5eadf9fa9e8ac52923a38d7 & Fixed case logic in SSTF\_dispatch
        \\ \hline
        Sat Oct 25 15:55 & Bob & b57104fba0b40f41f79c19d08310b543f8b7f6b9 & Added a hw2 Writeup doc.
        \\ \hline
        Sat Oct 25 21:36 & Bob & c918023c4d2f92f087e1cd721bde987b6291fe5a & Added Sam's project 2 writeup file.
        \\ \hline
        Sun Oct 26 21:57 & Bob & 4737476b348f7fb7acc3b47daa5328d41d393117 & Update sstf-iosched.c fully working.
        \\ \hline
        Mon Oct 27 02:38 & lawrencechau & 0fad02a3acef842d3d4b7ec6d4b1ebca2781cf29 & Update group\_22\_writeup.tex, Wrote the group writeup
        \\ \hline
        Mon Oct 27 11:53 & Bob & 50ea3799a520d50c9ee0f32ff3bf0f7b3b56a180 & Added the patch file with our SSTF I/O Scheduler.
        \\ \hline
        Mon Oct 27 11:55 & Bob & e06f0b76c25ebca39a0051d0132ba7c8ad74bf54 & Added the Linux Kernel source as a git ignored file.
        \\ \hline
        Mon Oct 27 11:56 & Bob & 43e34595c135bd9258a2b2ceaab1e0a069a41c8b & Added untracked files from previous concurrency problem.
        \\ \hline
    \end{tabular}
\end{center}
\end{document}
