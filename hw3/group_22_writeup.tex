\documentclass[letterpaper,10pt,notitlepage,fleqn]{article}

\usepackage{nopageno} %gets rid of page numbers
\usepackage{alltt}                                           
\usepackage{float}
\usepackage{color}
\usepackage{url}
\usepackage{balance}
\usepackage[TABBOTCAP, tight]{subfigure}
\usepackage{enumitem}
\usepackage{pstricks, pst-node}
\usepackage{geometry}
\geometry{textheight=9in, textwidth=6.5in} %sets 1" margins 
\newcommand{\cred}[1]{{\color{red}#1}} %command to change font to red
\newcommand{\cblue}[1]{{\color{blue}#1}} % ...blue
\usepackage{hyperref}
\usepackage{textcomp}
\usepackage{listings}

\def\name{Group 22}

\parindent = 0.0 in
\parskip = 0.2 in

\title{Project 2 Write Up}
\author{Group 22}

\begin{document}
\maketitle
\hrule

\section*{Our Plan}
    To create an encrypted \textit{RAM Disk} we decided to first get a unencrypted 
    version working before we begin to work with the \textit{Linux Crypto API}. 
    After reading over the assignment details and requirements we took a look at 
    the ``Linux Device Drivers'' example we discovered that finding a working example 
    of a \textit{RAM Disk} would not be too hard to find. After hearing about how 
    undocumented the \textit{Linux Crypto API} is in class we will begin our research 
    early. 

\section*{Our Solution}
    We took advantage of reusing the code from the ``Linux Device Drivers'' book that 
    was recommended for us to get used to programming Linux Drivers. We wanted to get 
    a working \textit{RAM Disk} working before we implemented our \textit{cipher}. 
    We added the sbd.c text that we grabbed from the book and after a few compile warnings 
    we found a blog by ``Pat Patterson'' that said he had fixed the warnings and errors 
    from the sbd.c file from the ``Linux Device Drivers''. Once we compiled the kernel with 
    the ramdisk included it was we needed to mount it. Using the command \textit{fdisk -l} 
    we discovered that our new \textit{RAM Disk} was mapped to \textit{/dev/sbd0} 
    since the \textit{RAM Disk} is just a unformated block of memory we decided to make a 
    ext2 filesystem and locally mounted it to a folder. After the Disk was mounted we copied 
    a few files to it and made sure it worked OK before we continued.\\
    Finding examples and any documentation for that mater was very difficult for the 
    \textit{Linux Crypto API} but after a bit of searching we eventually came to the conclusion 
    that you need to
    \begin{enumerate}
        \item Allocate crypto API
        \item Set cipher key
        \item Encrypt/Decrypt one byte at a time
        \item Free the cipher
    \end{enumerate}
    We attempted to use the ``Blowfish'' cipher algorithm instead of the standard ``AES'' 
    to try something different from the few examples we came across. It was hard to use the 
    \textit{Linux Crypto API} macros since they were fairly ambiguous. After a while of fiddling 
    we got our Kernel to compile and boot. Once the system started we formated the drive and mounted 
    it like we did before.\\
    To verify that our solution was correct we took advantage of the printk function and printed the 
    RAW data to the console before and after it was \textit{encrypted}. It was very easy to examine 
    the differences from before and after the \textit{encryption} took place. Even though we were printing 
    unsigned chars directly from the buffer and already couldn't read the data it was clear that something 
    changed when the \textit{encrypted} data was printed.

\section*{Work Log}

\begin{center}
    \begin{tabular}{| p{3cm} | l | l | p{5cm} |}
        \hline
        \textbf{Date} & \textbf{Author} & \textbf{Commit} & \textbf{Summary} \\ \hline
        Wed Nov 5 19:25 & Bob & 3f23e27044ff6f1d896075a36974c258c17a84f4 & Finnished Makefile and tar.bz2 for Concurrency 3.
        \\ \hline
        Wed Nov 5 19:21 & Bob & ce1c354c64c832cd137e17617b258c1c116034e0 & Finnised concurrency 3.
        \\ \hline
        Sun Nov 2 12:20 & Bob & 8feccdab21ba8750a0c5e9efe128aa44aa4fd3c4 & Merge branch 'master' of github.com:quinnsam/cs444
        \\ \hline
        Sun Nov 2 12:20 & Bob & 2cebcc999819ba670771c438eb14f813e99fc989 & Added hw3 folder and some documentation.
        \\ \hline
        Mon Nov 10 10:45 & Sam Quinn & b907ba3f829fae8178c7a05eb62e925eed699f56 & Added a working version of the encrypted Ramdisk.
        \\ \hline
        Mon Nov 10 10:49 & Bob & 3899e2b81125105ff1e1c6d3c08b6e7e4e8b152f & Added a mounting shell script to mount the encrypted ramdisk in one command.
        \\ \hline
        Mon Nov 10 10:53 & Bob & 3d5d384d6174a0667bb7c0b07e0bd21702b8151f &  Added original unecrypted ramdisk code from LLD3
        \\ \hline
        Mon Nov 10 10:56 & Bob & 3a9436d55efb23125f914d22181830a5e5f2d85d & Updated the .gitignore file.
        \\ \hline
        Mon Nov 10 11:01 & Bob & 93a01769fcd7d5cb670419a7c68baa4a91dda6a6 & Added the outline for Sam's Writeup.
        \\ \hline
        Mon Nov 10 18:08 & Bob & 689a1ac1ff20bd4d32c1f318984277c05cb8cf85 & Updated Sam's writeup with finished version.
        \\ \hline
        Mon Nov 10 18:09 & Bob & e9e015e98d3b8e04c2a8762b4af20029ce72245a & Added a pdf version of Sam;s Writeup.
        \\ \hline
        Mon Nov 10 18:13 & Bob & b3e07347b213607769bfcf75739449ecc6a4be62 & Added the tar ball for Sam's writeup.
        \\ \hline
        Mon Nov 10 18:27 & Bob & e744aa955ac823fb16abccbf23752374e396500f & Added the kernel patch file for the encrypted ram disk.
        \\ \hline
        Mon Nov 10 20:53 & Bob & cd1a064e295c87c3cafeee62842ca5b379735bc3 & Added the group Writeup tex and makefile
        \\ \hline
        Mon Nov 10 21:12 & Bob & af5c9f7eab1da3f867639174c34c4a1b97f2a394 & Finnished the group witeup.
        \\ \hline
    \end{tabular}
\end{center}
\end{document}









